% !TeX encoding = UTF-8
% !TeX spellcheck = hu_HU

\chapter{Saját munka bemutatása}

A fordítóprogramok nagyon komplex szoftverek. A legismertebb C/C++ fordítóprogram talán a GCC (GNU Compiler Collection), amely óriási kódbázissal rendelkezik, 2019-ben~körülbelül 15~millió sort tartalmaztak a forrásállományai \cite{GccWiki}. Természetesen egy GCC szintű fordítóprogram számos olyan funkcióval rendelkezik, mely a több, mint 36 évnyi fejlesztésből és a projekt léptékéből adódik, ilyen például a többféle programozási nyelv támogatása, valamint a fejlett optimalizációs megoldások. Ezen funkciók közül jó néhányat egy felhasználóifelület-leíró nyelveket támogató fordítóprogramnak nem szükséges biztosítania, és mivel az Önálló laboratórium tárgy választott témájának keretében a fordítóprogramok felépítésének és készítésének, valamint a programozási nyelvek konstrukcióinak mélyebb megismerése állt a középpontban, így elsősorban ezekre fektettem nagy hangsúlyt az irodalomkutatás során, továbbá ilyen területekre fogok összpontosítani a fordítóprogramom implementálása alatt is.



\section{Feladat specifikációja}
Az Önálló laboratórium projekt keretében egy úgynevezett \textit{source-to-source} fordítót készítek. Amint azt \az{ \sectref{compilerintro}}.~fejezetben ismertettem, az ilyen fordítók forrásfájlok közötti fordításra alkalmasak, azaz a bemenetként kapott fájlokból nem futtatható kódot generálnak, hanem egy másik nyelvre fordítják át azt. A projektfeladat keretében létrehozott fordító \az{\sectref{qtgtkoverview}}.~fejezetben és \az{\tabref{sourcecomparison}}~táblázatban bemutatott keretrendszerek felületleíró nyelveinek egy részhalmazát képes a QML, Qt XML és GTK XML között átalakítani.

\subsubsection{Támogatott formátumok}
A fentiek értelmében a program a tárgyalt két keretrendszer XML- és JSON-alapú nyelvei között teszi lehetővé a fordítást. Ennek megfelelően a bemeneti fájloknak az adott platform specifikációit kielégítően kell felépülniük, a kiterjesztésüknek pedig egyértelműen meghatározhatóvá kell tenni a forrásfájl típusát (azaz a \texttt{.ui} (Qt XML), \texttt{.qml} és \texttt{.glade} (GTK XML) hármas közül kell kikerülnie). A program ez alapján tudja eldönteni, hogy melyik értelmező (parser) modult kell meghívnia.

\subsection{Közös részhalmaz meghatározása} \label{sect:featuresubset}
Mivel a feladat során a hangsúly a fordítóprogramok felépítésének és működésének megismerésén van, továbbá az ismertetett keretrendszerek rengeteg felhasználásiterület-specifikus komponenssel rendelkeznek, melyeknek gyakran nincs is pontos megfelelője a másik keretrendszerben, ezért a fordítandó komponenseket az alapvető felhasználói felületi elemekre korlátoztam. Így végül egyszerű konténerek, gombok és beviteli mezők fordítására van lehetőség a programban.

A megfelelő részhalmaz kiválasztása a vártnál nagyobb kihívást jelentett, ugyanis sok felületi elem teljesen más tulajdonságokkal van ellátva a másik könyvtárhoz képest. Ilyen például, hogy GTK-ban alapvetően nem adható meg egy gomb pontos pozíciója és mérete, hanem kitölti a szülőkomponensben rendelkezésre álló helyet. Emiatt a gombok méretének beállítására csak akkor van lehetőség, ha a fordítás a két Qt-fájltípus, a \texttt{.ui} és a \texttt{.qml} között valósul meg (a fordítás iránya felcserélhető).

A programban lehetőség van a bemeneti fájl struktúrájának vizuális megjelenítésére is. Ehhez kimeneti fájlként egy \texttt{.svg} kiterjesztésű fájlt kell megadni. Ekkor a program az első két fázisban (ld. következő alfejezet) ugyanúgy viselkedik, mint általános esetben, de a kódgenerálási lépés során felületleíró fájl helyett egy vektorgrafikus állományt készít, mely egy irányított gráf formájában mutatja be a bemeneti fájl felépítését.

\section{A program felépítése és működése}
% TODO: dir ábra
\kep[0.67]{figures/dirtree.pdf}{A program architektúrája}{mycompilerarch}

Ahogy arról már \az{\sectref{compilerarch}} fejezetben is szó volt, a fordítóprogramok általában háromrétegűek, és a rétegek jól elkülöníthetőek. A munkám során törekedtem ezen architektúra követésére: külön egységbe kerültek a forrásnyelvelemző, a belső reprezentáció és a kódgenerálás forrásfájljai. Ezeken túl még röviden kitérek arra is, hogy a felhasználók hogyan használhatják a programot a parancssoron keresztül.


\subsection{Parancssori interfész}
A felhasználók parancssoron keresztül indíthatják el a programot, itt van lehetőség a forrás- és kimeneti fájlok elérési útjának megadására. A megadott paramétereket a Python által biztosított \texttt{ArgumentParser} osztály segítségével dolgozzuk fel. A megadott paraméterek helyességét is ellenőrizzük, ez látható az alábbi kódrészleten:

\begin{table}[h]
	\centering
	\begin{tabular}{c}
		\begin{minipage}[c]{0.6\linewidth}
			\begin{lstlisting}[numbers=left, frame=single, language=Python, caption={Hibaüzenetet előidéző kódrészlet}, captionpos=b]
  <item row="1" column="0">
    <widget class="QTimeEdit" name="timeEdit"/>
  </item>
			\end{lstlisting}
		\end{minipage}
	\end{tabular}
\end{table}

%TODO: Belső reprezentáció objektumorientált
% TODO: kép a tree-ről

\subsection{Front end}
A front end feladata, hogy ellenőrizze és értelmezze a kapott forrásfájlokat. A projektem során ide tartozó kódrészek a \textit{parsers} mappában kaptak helyet.

\subsection{Middle end}
Esetemben a middle end feladata egy egységes reprezentáció biztosítása a felhasználói felület elemei számára, hogy azok a forrás- és célplatformtól függetlenül legyenek kezelhetőek a programon belül.

\kep[1]{figures/uml.pdf}{A belső reprezentáció felépítése}{fig:ir}

\subsection{Back end}
Kódgenerálás


\subsection{Hibakezelés}
A program implementálása során nagy hangsúlyt fektettem a hibakezelésre. Ez igaz mind a felhasználói parancsok értelmezésére, mind a program belső működésére, ezzel is könnyítve a későbbi bővítést és növelve a hibakeresés hatékonyságát.


\begin{table}[h]
	\centering
	\begin{tabular}{c}
		\begin{minipage}[c]{0.6\linewidth}
			\begin{lstlisting}[numbers=left, frame=single, language=xml, caption={Hibaüzenetet előidéző kódrészlet}, captionpos=b]
  <item row="1" column="0">
    <widget class="QTimeEdit" name="timeEdit"/>
  </item>
			\end{lstlisting}
		\end{minipage}
	\end{tabular}
\end{table}

%Exceptionök, nem várt node-ok, programindítás paraméterei


% köztes reprezentáció a különböző bemenetek alapján a middle-endre
% xml, qml gráf rajzolása dotml schema (DOT) graphviz XML; svg generálása?

% gtk hw qt hw doxygen class hiersh

% szintaxisfa rajzolása

% generált felületekről képek

\section{Jelenlegi limitációk, továbbfejlesztési lehetőségek}
A program jelenleg a Qt és a GTK eszközkészletének csak egy kis részét fedi le, így könnyű olyan funkciót vagy felhasználóifelületi elemet találni, amelynek implementálásával érdemes lehet foglalkozni a jövőben.

Mivel mindkét könyvtár többféle megoldást kínál a vezérlőelemek felhasználói felületen való elrendezésére, ezért kézenfekvőnek tűnhet a különböző konténerosztályok feldolgozásának támogatását kitűzni következő mérföldkőként. Ehhez a motivációt növeli, hogy az összetettebb felhasználói felületek mind használnak ilyen osztályokat, általában többet is, egymásba ágyazva. A program jelenleg nem támogatja teljeskörűen az ilyen felületeket: GTK és Qt közötti fordításnál például bizonyos bemenetek esetén előfordulhat, hogy az elemek nem az elvárt pozícióba kerülnek. Ennek legfőbb oka, hogy a GTK-s vezérlők a Qt-nál látottakkal ellentétben alapértelmezetten kitöltik a számukra rendelkezésre álló helyet, és ezt csak különböző konténerek bevezetésével lehet felülírni, ami Qt-ban látottakhoz képest sokkal összetettebb logikát igényel a programban.

A másik fő irány, amiben a program továbbfejlesztésre szorul, az az elérhető vezérlők számának növelése. Ahogy arra \az{\sectref{featuresubset}} alfejezetben kitértem, a program jelenlegi állapotában csak néhány alapvető felhasználóifelületi elemet támogat, ami jelentősen korlátozza a felhasználási körét. Új vezérlők hozzáadása a program jelenlegi felépítése mellett már sokkal könnyebb, ugyanis már sok, az új elemek kezeléséhez szükséges megoldásra szükség volt a jelenlegi állapot eléréséhez is, ezek pedig a program moduláris mivoltából adódóan könnyen újrafelhasználhatóak. Ilyen megoldásnak tekinthető például a belső reprezentáció alaposztálya, az \texttt{AstBase}, melyből egy új osztályt leszármaztatva vehetjük fel a kívánt felületi elemet a programba. Emellett rendelkezésre állnak már egyes adatstruktúrák és fabejáró függvények a jövőbeni munka megkönnyítésére.
